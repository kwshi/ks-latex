\documentclass{ks}

\title{Precalculus worksheet}
\subtitle{Math tutoring stuff}
\author{Kye Shi}
\date{2021 September}

\usepackage{lipsum}

\begin{document}

\begin{enumerate}

  \item Consider a parabola defined by \(y = f(x) = ax^2 + b\).  On the plot of
    this parabola, mark two points at \((x_1, f(x_1))\) and \((x_2, f(x_2))\),
    where \(x_1 \ne x_2\).

    \begin{center}
      \begin{tikzpicture}
        \draw[<->, annot] (-1in-1em,-1em) -- (1in+1em,-1em);
        \draw[<->, annot] (0,-2em) -- (0,1in+1em);
        \draw (-1in,1in) parabola[bend at end] (0,0) parabola (1in,1in);
        \coordinate[dot](a) at (-1in/2,1in/4);
        \coordinate[dot](b) at (3in/4,9in/16);
        \node[above left] at (a) {\((x_1, f(x_1))\)};
        \node[below right] at (b) {\((x_2, f(x_2))\)};
        \draw[thick]
          ($ (a) + -2/5*($ (b)-(a) $) $)
          -- ($ (b) + 1/5*($ (b)-(a) $) $);
      \end{tikzpicture}
    \end{center}

    \begin{enumerate}
      \item What is the slope of line passing through these two points?
        (Simplify your result as much as you can.)
      \item What happens to the slope if \(x_1\) and \(x_2\) are really close
        to each other?
    \end{enumerate}

  \item Look at the following plots of functions.  Which ones would you consider
    are \emph{continuous} (you do not need to know the official definition of
    ``continuous'' for this)?  Which ones would you call ``smooth''?

    Be more specific---for the functions that \emph{aren't} continuous and/or
    smooth, can you say \emph{where} they are discontinuous and/or not-smooth?
    Where \emph{are} they continuous and/or smooth?

    \tikzset{
      mini/.style={
        baseline={(current bounding box.center)},
        /pgfplots/.cd,
        xtick=\empty,
        ytick=\empty,
        width=3in,
        height=2in,
      },
    }
    \pgfplotsset{
      mini/.style={
        xtick=\empty, ytick=\empty, width=3in, height=2in,
      },
    }

    \begin{multicols}{2}
      \begin{enumerate}
        \item \begin{tikzpicture}[mini]
            \begin{axis}
              \addplot[domain=-4*pi:4*pi, samples=256]{sin(x r)/x};
            \end{axis}
          \end{tikzpicture}
        \item \begin{tikzpicture}[mini]
            \begin{axis}
              \addplot[domain=-4:4, samples=5]{abs(x)};
            \end{axis}
          \end{tikzpicture}
        \item \begin{tikzpicture}[mini]
            \begin{axis}
              \addplot[domain=-pi/4:pi/4, samples=16384, line join=round]{sin(1r/x)};
            \end{axis}
          \end{tikzpicture}
        \item \begin{tikzpicture}[mini]
            \begin{axis}
              \addplot[domain=-1:0, samples=64]{1/x};
              \addplot[domain=0:1, samples=64]{1/x};
              \coordinate[dot]() at (0,0);
            \end{axis}
          \end{tikzpicture}
      \end{enumerate}
    \end{multicols}

  \item How is your trigonometry?  Use trig-function identities to simplify the
    following as best as you can:

    \begin{enumerate}
      \item \(\frac 1 8 (3 + \cos 4x - 4 \cos 2x)\).
      \item \(\cos \frac x 4\).  Use your answer to evaluate the exact value of
        \(\cos \frac \pi {16}\).
    \end{enumerate}

  \item Assume \(\vec i^2 = -1\).
    \begin{enumerate}
      \item \label{itm:angle-sum} Use trig-identities to simplify:
        \[
          (\cos \theta + \vec i \sin \theta) (\cos \phi + \vec i \sin \phi).
        \]
      \item What does your result from \ref{itm:angle-sum} imply about
        \[
          (\cos \theta + \vec i \sin \theta)^n
        \]
        (for positive integers \(n\))?
    \end{enumerate}



  \item What is the inverse function of \(f(t) = c t^{c-1}\)?

  \item
    \begin{enumerate}
      \item Expand and simplify \((x+y)(x-y)\).
      \item Expand \((x+y)(x^2-xy+y^2)\).
      \item What pattern do you observe?  Fill in the blank and prove: \emph{For
        every positive integer \(n\)},
        \[
          (x+y)
          (\underline{\phantom{x^n - x^{n-1} y + x^{n-2} y^2 - \dotsb \pm y^n}})
          = \underline{\phantom{x^{n+1} - y^{n+1}}}.
        \]

    \end{enumerate}

  \item Exponents rule: exponent rules!  Fill in the following:
    \begin{multicols}{3}
      \begin{enumerate}
        \item \(a^x a^y = \underline{\mathstrut\qquad}\).
        \item \((b^x)^c = \underline{\mathstrut\qquad}\).
        \item \(x^{\log_x y} = \underline{\mathstrut\qquad}\).
        \item \(\log_a b \cdot \log_b c = \underline{\phantom{\log_a c}}\).
        \item \(\log_k xy = \underline{\phantom{\log_k x + \log_k y}}\).
        \item \(\log_c x^p = \underline{\phantom{p \log_c x}}\).
      \end{enumerate}
    \end{multicols}

\end{enumerate}

\end{document}
